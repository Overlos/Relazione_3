\subsection{e su m}
A partire dai dati raccolti con la procedura descritta in [ref paragrafo] è stato calcolato il rapporto $\frac{e}{m}$ e l'incertezza sulle misure con errore statistico è stata propagata col metodo delle derivate parziali. Dai 15 valori cosi' ottenuti è stata calcolata la media pesata con relativa incertezza e a questa è stato sommato in quadratura l'errore sistematico dovuto al raggio della traiettoria compiuta dagli elettroni. 
%tabelle media
È stato fatto anche un fit lineare usando $2V$ come ascissa e $(B_rR)^2$ come ordinata. L'incertezza su $2v$ è stata considerata trascurabile.\footnote{È stato eseguito anche un fit in cui l'incertezza su $2V$ è stata propagata usando una prima approssimazione della linea di tendenza a partire da tre misure di test. Il risultato cosi' ottenuto non è riultato essere significativamente diverso da quello ottenuto col metodo presentato.}
Per visionare i dati relativi alla regressione lineare si rimanda a [ref appendice].
%tabelle fit

La compatibilità dei dati col risultato fornito dal fit lineare è stata valutata usando il test del $\chi^2$.

\subsection{B terra}
A partire dai  dati raccolti usando la procedura descritta in [ref paragrafo] è stata calcolata la componente trasversale (??) del campo magnetico terrestre. L'incertezza è stata propagata col metodo delle derivate parziali. Di questi risultati è stata effettuata una media pesata.
%tabelle B_t


