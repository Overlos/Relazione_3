\subsection{Cenni storici}
Nel 1896, il fisico britannico J. J. Thomson, con i suoi colleghi John S. Townsend e H. A. Wilson, svolsero una serie di esperimenti che dimostrarono che i raggi catodici erano costituiti da singole particelle, piuttosto che onde, atomi o molecole come si riteneva in precedenza.Thomson stimò in maniera accurata la carica e la massa, trovando che le particelle dei raggi catodici, che lui chiamò "corpuscoli", avevano probabilmente una massa migliaia di volte inferiore a quella dell'idrogenione (H+), lo ione più leggero che si conoscesse a quel tempo.Thomson mostrò come il rapporto carica/massa (e/m), uguale a $5,273 \times 10^17$  e/g, fosse indipendente dal materiale del catodo. Inoltre mostrò come le particelle cariche negativamente prodotte dai materiali radioattivi, dai materiali riscaldati e dai raggi catodici fossero riconducibili tutte alla stessa entità.
