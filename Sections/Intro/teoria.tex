\subsection{Cenni teorici}
%Wenn die Soldaten

La misura del rapporto $\e$ su $m$ si basa sull'osservazione della traiettoria di elettroni eccitati da una differenza di potenziale $\Delta\volt$ presente fra il catodo e l'anodo di un cannone elettronico posto in un'ampolla contente idrogeno ad una pressione di circa $10^{-2}\,torr$.
Gli atomi di $H$, se eccitati decadono in un tempo brevissimo emettendo fotoni nella lunghezza d'onda di $\sim 450\nano\meter$.
Questi elettroni vengono deflessi, una volta emessi, dal campo magnetico generato da due bobine di Helmholtz che li obbliga a percorrere una traiettoria circolare, della quale è possibile misurare il raggio.
L'ampolla è posizionata nel centro della coppia di bobine.
Chiamiamo ora $I$ l'intensità della corrente necessaria ad indurre il campo magnetico $B_z$ delle due bobine, $N$ il numero di spire e con $R_b$ il raggio medio dell'ampolla
L'intensità del campo è data da 
\begin{center}
	\begin{equation*}
	B_z(0)=\mu_0\frac{8}{5\sqrt5}\frac{NI}{R_b}
	\end{equation*}
\end{center}


